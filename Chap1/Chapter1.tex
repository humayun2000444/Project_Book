\chapter{Introduction}

\section{Overview}
Online examinations have become integral to modern education, particularly accelerated by the global shift toward remote learning. However, maintaining academic integrity in virtual environments presents significant challenges. This project develops an AI-powered online exam proctoring platform leveraging YOLOv8n for object detection and MediaPipe for facial analysis, enabling automated real-time monitoring. The dual-camera architecture using WebRTC technology captures desktop and mobile feeds simultaneously, providing multi-perspective surveillance. Real-time alerts via Socket.IO ensure $<$150ms notification latency. Built on React.js, Flask, and MySQL, the platform provides end-to-end examination management with automated MCQ grading, manual CQ grading, violation tracking with automatic banning, and compliance with GDPR, CCPA, and FERPA.

\section{Problem Statement}
Traditional examination systems face critical limitations: \textbf{Manual Proctoring Constraints} (single proctor limited to 8-10 students before accuracy drops below 60\%, prohibitively expensive for large-scale exams), \textbf{Inconsistent Monitoring} (cognitive fatigue reduces detection accuracy by 23-35\% after 45 minutes), \textbf{Limited Environmental Awareness} (single webcam misses 42\% of violations occurring outside field of view), \textbf{Delayed Response} (post-exam review allows ongoing violations), \textbf{Privacy Concerns} (34\% of systems lack end-to-end encryption), and \textbf{High Costs} (\$15-\$30 per student makes frequent assessments economically unfeasible). These limitations necessitate an automated, scalable, cost-effective solution with real-time detection, comprehensive surveillance, and robust privacy protections.

\section{Motivation}
Development is motivated by: \textbf{Global Shift to Remote Learning} (1.6 billion students affected by COVID-19 closures requiring reliable remote assessments), \textbf{AI Advancement} (YOLOv8n achieves 6-7 FPS on CPU-only systems, eliminating GPU infrastructure costs), \textbf{Scalability} (unlimited concurrent sessions with consistent accuracy), \textbf{Economic Benefit} (82\% cost reduction from \$250,000 to \$45,800 annually), \textbf{Academic Integrity} (68\% witness online cheating vs. 43\% in-person), and \textbf{Privacy Compliance} (GDPR/CCPA/FERPA require privacy-by-design with data minimization and encryption).

\section{Objectives}

Develop a comprehensive AI-powered proctoring platform with the following objectives: \textbf{(1) Real-Time Proctoring:} YOLOv8n object detection, MediaPipe facial recognition, WebRTC streaming, Socket.IO communication. \textbf{(2) Dual-Camera Monitoring:} Desktop webcam and mobile camera (QR code pairing) for comprehensive surveillance. \textbf{(3) Violation Detection:} Multiple person detection, electronic device identification, prohibited object detection, window switching monitoring, identity verification. \textbf{(4) Exam Management:} MCQ auto-grading, CQ manual grading, file submissions, A+ to F grading scale, automatic banning at 5 violations. \textbf{(5) Notifications:} Socket.IO push alerts $<$150ms, 10-minute pre-exam reminders, department/batch/section filtering. \textbf{(6) RBAC Authentication:} JWT-based three-role system (Student, Teacher, Administrator). \textbf{(7) Security:} HTTPS/TLS encryption, biometric data protection, GDPR/CCPA/FERPA compliance, audit logging. \textbf{(8) Architecture:} Flask RESTful API, React.js/Tailwind CSS frontend, MySQL database, CPU-optimized AI models. \textbf{(9) Analytics:} Timestamped violation logs, performance dashboards, live statistics, screenshot evidence reports. \textbf{(10) Extensibility:} RESTful API for third-party integration, modular components, scalable database schema.

\section{Scope of the Project}

Project scope encompasses: \textbf{System Development} (full-stack React.js/Flask application, YOLOv8n/MediaPipe integration, dual-camera WebRTC streaming, Socket.IO notifications, exam management workflows), \textbf{Security} (JWT authentication, RBAC, HTTPS/TLS and AES-256 encryption, GDPR/CCPA/FERPA compliance, audit logging), \textbf{Testing} (unit testing, AI accuracy validation, performance testing, security penetration testing, user acceptance testing), and \textbf{Documentation} (system architecture, API documentation, database schema, user manuals, deployment guides).

\section{Limitations and Constraints}

Platform limitations include: \textbf{Hardware Dependencies} (requires webcam and stable internet, creating equity concerns), \textbf{Processing Performance} (6-7 FPS on CPU with 150-200ms detection delay may miss rapid violations), \textbf{False Positives} (requires manual proctor review to prevent unjust penalties), \textbf{Privacy Concerns} (biometric data collection despite encryption and compliance measures), and \textbf{Cultural/Accessibility Factors} (behavioral norms vary across cultures, disabilities may trigger false violations).

\section{Organization of the Report}

\textbf{Chapter 2:} Background study with literature review, comparative analysis of commercial proctoring platforms, and technical/operational/economic/legal/market feasibility studies. \textbf{Chapter 3:} System design with flowcharts, workflow diagrams, use case diagrams, activity diagrams, sequence diagrams, data flow diagrams (Levels 0/1/2), and entity-relationship diagrams. \textbf{Chapter 4:} Implementation details including technology stack, API endpoints, database schema, AI model integration, and real-time communication architecture. \textbf{Chapter 5:} Results and testing with performance metrics, AI detection accuracy, scalability testing, security assessment, and user acceptance outcomes.
