The shift toward online education has created an urgent need for reliable exam proctoring solutions. While traditional manual proctoring works well in physical classrooms, it doesn't scale economically for remote assessments—costs become prohibitive, and human proctors struggle to monitor multiple video feeds simultaneously. This research addresses these limitations by developing an AI-powered proctoring platform that monitors exams in real-time without human intervention.

Our approach combines two complementary AI technologies: YOLOv8n detects physical objects like mobile phones or unauthorized materials, while MediaPipe tracks facial movements and gaze direction. What makes this system unique is the dual-camera architecture. We discovered during early testing that a single webcam provides an incomplete view of the exam environment. By pairing the student's desktop camera with their mobile phone camera (connected via QR code), we achieve comprehensive room monitoring without requiring specialized hardware. The entire system communicates through WebRTC streaming, with Socket.IO delivering violation alerts to instructors in under 150ms.

Beyond monitoring, the platform handles complete exam workflows. Multiple-choice questions are graded automatically. Constructed-response questions flow to a manual grading interface. Students who accumulate five violations get banned automatically—though instructors can review and reverse decisions through an appeals process. We built the frontend with React.js and Tailwind CSS, backed by a Flask REST API and MySQL database. Security came first: JWT tokens control access, HTTPS/TLS encrypts all traffic, and we implemented GDPR, CCPA, and FERPA compliance from the ground up.

When we compared costs against traditional proctoring for 10,000 annual exams, the platform reduced expenses by 82\%—paying for itself in just over four months. More importantly, it scales without adding staff.

\vspace{8pt}
\textbf{Keywords:} Online Exam Proctoring, Artificial Intelligence, YOLOv8, MediaPipe, Computer Vision, Real-time Monitoring, WebRTC, Academic Integrity.
